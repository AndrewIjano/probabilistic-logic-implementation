\documentclass[portrait]{ppgcaposter}
\usepackage{lmodern} 

\usepackage{mdframed}
\usepackage{xcolor}

\definecolor{navyblue}{rgb}{0.0, 0.0, 0.5}

% EL
\newcommand{\el}{\mathcal{EL}} 
% EL++
\newcommand{\elpp}{\mathcal{EL}^{++}}
% GEL++
\newcommand{\gelpp}{\mathcal{GEL}^{++}}
% concrete domain D
\newcommand{\D}{\mathcal{D}}
% concept names
\newcommand{\Nc}{\mathsf{N_C}}
% role names
\newcommand{\Nr}{\mathsf{N_R}}
% individual names
\newcommand{\Ni}{\mathsf{N_I}}
% interpretation
\newcommand{\I}{\mathcal{I}}

\usepackage{tikz}
\usetikzlibrary{shapes,arrows, positioning}

\setlength{\parskip}{0em}

\mdfdefinestyle{MyFrame}{%
    backgroundcolor=navyblue,
    fontcolor=white,
    linecolor=navyblue,
    % outerlinewidth=2pt,
    %roundcorner=20pt,
    innertopmargin=30pt,
    innerbottommargin=30pt,
    innerrightmargin=4pt,
    innerleftmargin=4pt,
        leftmargin = 4pt,
        rightmargin = 4pt
}

\DeclareCaptionFont{small}{\small}
\captionsetup{font=small}


\begin{document}
\fontfamily{lmss}\selectfont
\renewcommand{\familydefault}{\sfdefault}
% \printheader
\renewcommand{\section}[2]{}

\begin{center}
    %% TITLE
    \textbf{\bf\veryHuge\color{NavyBlue}\fontfamily{lmss}\selectfont Raciocínio Probabilístico Tratável em um Fragmento da Lógica de Descrição\\[1.5cm]}
    \Huge \text{MAC0215 - Atividade Curricular em Pesquisa}\\ [0.8cm]
    %% AUTHORS
    \huge \textbf{Andrew Ijano Lopes}\\ [0.8cm]
    \huge Orientador: Marcelo Finger\\ [0.2cm]
    \huge Instituto de Matemática e Estatística da Universidade de São Paulo
\end{center}

\vspace{1cm}

\Large

\begin{multicols}{2} % begin two columns

    \color{black}

    % \section*{Introdução}

    \begin{mdframed}[style=MyFrame]
        \begin{center}
            \huge  \textbf{Introdução}
        \end{center}
    \end{mdframed}
    \paragraph{}
    As lógicas de descrição são uma família de linguagens formais para a representação do conhecimento. Dentre as várias propostas existentes, a lógica $\elpp$ se mostrou uma das lógicas de descrição mais expressivas cuja complexidade de inferência é decidível em tempo polinomial.~\cite{Baader2005a} Porém, ao adicionar probabilidade às capacidades de inferência seu problema de decisão se torna NP-Completo.~\cite{Fin2019b}
    \paragraph{}
    Com um estudo mais recente, está sendo proposto que, ao restringir a expressividade do $\elpp$, removendo a conjunção, é possível construir uma lógica de descrição probabilística tratável.
    \paragraph{}
    Nesse projeto, será usado a linguagem \emph{Graphic} $\elpp$ ($\gelpp$), uma restrição da linguagem $\elpp$, removendo as conjunções.
    \\
    \begin{center}
        \begin{minipage}{0,2\textwidth}
            \begin{description}
                \fontsize{28}{28}
                \selectfont
                \item Fever $\sqsubseteq$ Symptom
                \item FeverAndRash $\sqsubseteq$ Fever
                \item Dengue $\sqsubseteq$ Disease
                \item Symptom $\sqsubseteq \, \exists$hasCause.Disease
                \item Patient $\sqsubseteq\, \exists$hasSymptom.Symptom
                \item hasSymptom $\circ$ hasCause $\sqsubseteq$ suspectOf
            \end{description}
        \end{minipage}
        \hfill \vline
        \hfill
        \begin{minipage}{0,2\textwidth}
            \fontsize{28}{28}
            \selectfont
            \begin{description}
                \item $\{$john$\}$ $\sqsubseteq$ Patient
                \item $\{$s1$\}$ $\sqsubseteq$ FeverAndRash
                \item \{john\} $\sqsubseteq \, \exists$hasSymptom.\{s1\}
                \item DenguePatient $\equiv\, \exists$suspectOf.Dengue
                      % \item hasSymptom $\circ$ hasSymptom $\sqsubseteq$ suspectOf
            \end{description}
        \end{minipage}
    \end{center}
    \captionof{figure}{Exemplo da representação de um caso médico com uma ontologia em $\gelpp$.}
    \vspace{60pt}
    \begin{mdframed}[style=MyFrame]
        \begin{center}
            \huge  \textbf{Objetivos}
        \end{center}
    \end{mdframed}

    \paragraph{}
    O objetivo da pesquisa é realizar estudos sobre diversos conceitos para conseguir propor um algoritmo de satisfatibilidade, estudar sua complexidade e implementar esse algoritmo tratável para um fragmento de uma lógica de descrição probabilística.
    \paragraph{}
    Para o escopo dessa disciplina, o projeto abrange a implementação do início do algoritmo sem a inclusão de probabilidades: leitura da ontologia e a resolução do problema de satisfatibilidade máxima ($\gelpp$-\emph{MaxSAT}). Esse algoritmo será aplicado na linguagem $\gelpp$.
    \\
    \begin{mdframed}[style=MyFrame]
        \begin{center}
            \huge  \textbf{Metodologia}
        \end{center}
    \end{mdframed}
    \paragraph{}
    Na primeira etapa do projeto, foi implementado a identificação e leitura de uma ontologia na linguagem $\gelpp$, codificada no formato OWL. Esse processo envolve observar se a ontologia se enquadra na expressividade do $\gelpp$ e, em caso afirmativo, analisar cada componente, gerando o grafo correspondente. Para essa etapa, foi utilizada a biblioteca OWL API~\cite{OWLAPI}, em Java.
    \paragraph{}
    Em seguida, foi implementado uma interface entre o programa de leitura, em Java, com o algoritmo de satisfatibilidade, em \texttt{C++}, usando a \emph{Java Native Interface} (JNI) ~\cite{JNI}.
    Além disso, foi implementada a estrutura de dados equivalente ao grafo em Java para armazenar a ontologia e realizar o algoritmo.
    \paragraph{}
    Ainda, foi feito o início do algoritmo de cortes mínimos. Um segundo modelo de grafos, apenas com pesos nas arestas, foi implementado e está sendo estudado as aplicações das técnicas de fluxo máximo e corte mínimo no grafo. Disso, a implementação do $\gelpp$-\emph{MaxSATSolver}, algoritmo que resolve o problema de satisfatibilidade máxima, é facilmente produzida.
    \columnbreak
    \begin{mdframed}[style=MyFrame]
        \begin{center}
            \huge  \textbf{Discussões\vphantom{ç}}
        \end{center}
    \end{mdframed}

    \paragraph{}
    \tikzstyle{block} = [draw, rectangle, rounded corners=.55cm, fill=black!20, draw=white]
    \tikzstyle{dashblock} = [draw, rectangle, dashed, rounded corners=.55cm, fill=black!10, draw=black!30, line width=5pt]

    \tikzstyle{arrow} = [>=to, line width=3pt]
    \begin{center}
        \begin{tikzpicture}[auto, node distance=2cm, ->, >=latex]
            \node (init) {Init};
            \node [below=of init] (s1) {$\{$s1$\}$};
            \node [below=of s1] (john) {$\{$john$\}$};

            \node [block, right=of s1] (fevrash) {FeverAndRash};
            \node [block, right=of fevrash] (fever) {Fever};
            \node [block, right=of john] (patient) {Patient};
            \node [block, above=of fever] (dengue) {Dengue};
            \node [block, above=of dengue] (denguepat) {DenguePatient};
            \node [block, right=of dengue] (disease) {Disease};
            \node [block, right=of fever] (symptom) {Symptom};
            \node [dashblock, left=of denguepat] (suspofden) {$"$suspectOfDengue$"$};
            \node [red, left=of dengue] (bot) {$\bot$};

            \path [arrow]
            (init) edge (s1)
            (init) edge[bend right = 50] (john)
            (s1) edge (fevrash)
            (john) edge (patient)
            (dengue) edge (disease)
            (fevrash) edge (fever)
            (fever) edge (symptom)
            (suspofden) edge (denguepat)
            ;

            \path [arrow, dotted, blue]
            (john) edge node [right] {hasSymptom} (s1)
            (patient) edge [bend right = 20] node [midway, below] {hasSymptom} (symptom)
            (denguepat) edge node [right] {suspectOf} (dengue)
            (symptom) edge node [right] {hasCause} (disease)
            ;

            \path [arrow, dotted, red]
            (fevrash) edge node [left] {hasCause} (dengue)
            (fever) edge node [right] {hasCause} (dengue)
            ;

            \path [arrow, red]
            (dengue) edge (bot)
            ;

        \end{tikzpicture}
        \captionof{figure}{Exemplo da representação de uma ontologia em $\gelpp$ em grafo. Arcos contínuos correspondem à subsunção e os rótulos de arcos pontilhados são propriedades. Numa implementação com probabilidades, os arcos vermelhos indicam que o arco tem probabilidade.}
    \end{center}

    \paragraph{}
    Como citado nos objetivos do projeto, essa pesquisa aplicou métodos para leitura de ontologias em $\gelpp$, implementou estruturas de dados para armazenar ontologias dessa família de linguagens e resolver problemas de corte mínimo em grafos com pesos. Dentre os objetivos previstos, a resolução do problema de $\gelpp$-\emph{MaxSAT} ainda está em andamento e está sendo implementada junto com o problema de corte mínimo.
    \paragraph{}
    Durante o desenvolvimento do projeto, foi despendido um tempo muito maior que o esperado na etapa de leitura da linguagem. Uma das maiores problemáticas dessa parte foi a falta de documentação atualizada sobre a OWL API, aumentando o tempo de aprendizagem no uso da biblioteca.
    \paragraph{}
    Para fora do escopo da disciplina, o projeto ainda visa o estudo da representação de probabilidades em lógicas de descrição e a implementação de um algoritmo de satisfatibilidade para o $\gelpp$ probabilístico utilizando o $\gelpp$-\emph{MaxSATSolver}.
    \\
    \\
    \begin{mdframed}[style=MyFrame]
        \begin{center}
            \huge  \textbf{Referências\vphantom{ç}}
        \end{center}
    \end{mdframed}
    \bibliographystyle{plain} % Plain referencing style
    \paragraph*{}
    \bibliography{refs}
    % \includegraphics[width=0.25\linewidth]{ime-logo.svg.png}
\end{multicols}
\end{document}